\documentclass[modern]{aastex63}
\usepackage[utf8]{inputenc} 
\usepackage{hyperref}
\usepackage{listings}
\usepackage{fancyhdr}   
\usepackage{geometry}
\usepackage{amsmath}
\renewcommand{\baselinestretch}{1.0}
\usepackage{amssymb}
%\renewcommand{rnaas}{True}

\geometry{lmargin=1in, bmargin=1in, tmargin=0.5in, rmargin=1in}

\pagestyle{plain}

\begin{document}
%\title{Research Statement}
\title{Disentangling Sources of SN Ia Variation}

\author{Jared Hand}
\affiliation{University of Pittsburgh, \textnormal{jsh89@pitt.edu}}
%\affiliation{\textbf{\textnormal{jsh89@pitt.edu}}}

% \begin{abstract}
%     We ask to supplement a {\bf graduate student} and fund an {\bf undergraduate} in a project studying the influence of dust attenuation correction on SN~Ia host bias corrections and its connection to SN~Ia color variation.
%     Better understanding SN~Ia color variation and characteristics of the host bias will improve the utility of SNe~Ia for measuring dark energy in the era of the Vera C. Rubin Observatory LSST survey.\\
%     {\bf Request: \$10,000}
% \end{abstract}
\section{Introduction}
Resolving outstanding issues standardizing Type Ia supernova (SN~Ia) for use as cosmic distance measures is the focus dissertation at the University of Pittsburgh. Host galaxy dust attenuation is coupled to both the observed time-independent (color) variation in SN~Ia populations and an established bias between SN~Ia and host galaxy properties, neither of which are appropriately accounted for during lightcurve standardization. I have demonstrated the consistency of the host mass step bias measurement regardless of chosen observation or fitting technique \citep{Hand2021}. This coming year I will determine both the dependence of star formation rate (SFR) tracers and dust attenuation technique on the measured host SFR step bias. Concurrently, with Dr. Alex Kim of the Lawrence Berkeley National Laboratory we have developed a model separating intrinsic SN~Ia color variation from dust effects. We will develop this model further into a fully generative, high resolution SN~Ia model to improve SN~Ia standardization.  Together, these two projects will decouple sources of SN~Ia color variation to better standardize SN~Ia and appropriately account for the outstanding systematics from the SN~Ia host bias, notably increasing SN~Ia utility in cosmology.

\section{Research Statement}
The extreme brightness of SNe~Ia are standardized using correlations between peak brightness and both SN~Ia `blueness' and explosion duration to reduce the already naturally low scatter in peak brightness, allowing their use as cosmic `rulers' to constrain cosmological models and explore the properties of dark energy \citep{Perlmutter1999}.
A lack of theoretical understanding SN~Ia necessitates these transient objects be parameterized by generative empirical models trained from representative SN samples.
The current linear Tripp standardization technique and commonly used SN~Ia models cannot differentiate between time-independent wavelength (color) variation intrinsic to the SN~Ia population and color variation from host galaxy dust attenuation \citep{Mandel2017}, and overcorrect for the established SN~Ia host bias where brighter SN~Ia prefer younger, less massive hosts with higher SFR and younger stellar populations \citep{Sullivan2010,Rigault2018}.
SFR and stellar age are highly correlated with both each other and dust attenuation
This host bias propagating through Tripp standardization increases cosmological measurement uncertainty and is driven both by mishandling of dust effects on SN~Ia color variation and ignoring the established host bias during standardization \citep{Brout2021,Rose2021}.

\subsection{Determining the Connection between Dust and the Host Bias}
%{\bf Determining the Connection between Dust and the Host Bias:}
My first paper, recently resubmitted after review, compares observation and fitting techniques used to measure the SN~Ia host bias and alleviates concern that such choice in technique influences mass step measurements; it remains unclear if this is true for SFR and stellar age step bias measurements \citep{Hand2021}.  My current work expanding upon this first paper is quantifying any dependence of host SFR and stellar~age on host dust attenuation measurement using the observational SFR tracers H$\alpha$ emission, which provides an instantaneous ($<10$~Myr) SFR measurement, and UV flux, which provides an extended ($10-100$~Myr) SFR measurement.
UV and H$\alpha$ vary dramatically in their susceptibility to dust attenuation, providing independent probes of host dust content.
We will integrate sophisticated galaxy attenuation curve modeling from~\cite{Salim2018} and~\cite{Narayanan2018} in a notable improvement from past SN~Ia analyses that will also increase quality of UV flux dust corrections.
Numerous covariates including latent model parameters intrinsic to SN~Ia standardization necessitates efficient posterior analysis which will be performed using a hierarchical Bayesian framework implemented with a Hamiltonian Monte Carlo sampler built into the software package Stan~\citep{STAN}.  This continued work will produce another paper within the next two years.

\subsection{Separating Sources of SN~Ia Color Variation}
%{\bf Separating Sources of SN~Ia Color Variation:}
As part of a DOE SCGSR award I received in 2019 to work with Dr.~Alex Kim, we used spectroscopic SN~Ia time series to build a hierarchical Bayesian model with Gaussian processes to agnostically disentangle the intrinsic color variation of SNe~Ia from host dust effects.
Parameterizing color variation using a set of multiplicative color variation curves that makes no strong prior assumption on color curve shape, our model exploits mathematical symmetries to simultaneously recover dust attenuation effects and intrinsic color variation.
Already, we have developed the mathematical and software foundations for this complex model and found evidence for intrinsic color variation independent from variation due to dust.
I am preparing our paper presenting these initial results.
Over the next year we will fully develop our model into a fully descriptive SN~Ia empirical model by adding parameterized time-dependent color variation templates, providing a model that completely separately parameterizes all sources of color variation. 
We next will promote our model to a proper Gaussian mixture model and simultaneously classify SN~Ia subtype and parametrically fit light curves.  
A second paper (and fourth overall) will present our full model and its utility in SN~Ia standardization.\\

{\bf Mentoring Undergraduate Researchers:}
Joining the host bias analysis project are two undergraduates of the Michael Wood-Vasey research group who I will mentor.  This will include preparing to present their work to the Dark Energy Science collaboration and guidance in manuscript preparation and writing.  I strongly believe early hands-on exposure to data manipulation, analysis, and manuscript preparation are crucial for undergraduates intent on furthering their education through graduate school and am excited to guide these students throughout my remaining years as a PhD student.  

{\bf Dissertation Progress Notes:}
My second committee meeting and DOE award were both delayed by health issues during 2020 and the COVID-19 pandemic. 
The second meeting is now planned for June 2021 and the DOE award, restarted remotely in January 2021, will conclude July 2021. 

\newpage
\bibliographystyle{aasjournal}
\bibliography{main}

\end{document}
