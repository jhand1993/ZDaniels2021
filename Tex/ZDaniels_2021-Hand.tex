\documentclass[modern]{aastex63}
\usepackage[utf8]{inputenc} 
\usepackage{hyperref}
\usepackage{listings}
\usepackage{fancyhdr}   
\usepackage{geometry}
\usepackage{amsmath}
\renewcommand{\baselinestretch}{1.0}
\usepackage{amssymb}
%\renewcommand{rnaas}{True}

\geometry{lmargin=1in, bmargin=1in, tmargin=0.5in, rmargin=1in}

\pagestyle{plain}

\begin{document}
%\title{Research Statement}
\title{Disentangling Sources of SN Ia Variation}

\author{Jared Hand}
\affiliation{University of Pittsburgh, \textnormal{jsh89@pitt.edu}}
%\affiliation{\textbf{\textnormal{jsh89@pitt.edu}}}

% \begin{abstract}
%     We ask to supplement a {\bf graduate student} and fund an {\bf undergraduate} in a project studying the influence of dust attenuation correction on SN~Ia host bias corrections and its connection to SN~Ia color variation.
%     Better understanding SN~Ia color variation and characteristics of the host bias will improve the utility of SNe~Ia for measuring dark energy in the era of the Vera C. Rubin Observatory LSST survey.\\
%     {\bf Request: \$10,000}
% \end{abstract}
\section{Introduction}
Resolving outstanding issues standardizing Type Ia supernova (SN~Ia) for use as cosmic distance measures will be the focus of my fifth year as a physics graduate student at the University of Pittsburgh. Host galaxy dust attenuation is coupled to both the observed time-independent (color) variation in SN~Ia populations and an established bias between SN~Ia and host galaxy properties, neither of which are appropriately accounted for during lightcurve standardization. Resolution requires using hierarchical modeling to separate variation from dust from that intrinsic SN~Ia and quantifying attenuation's contribution to the observed host bias by star formation observational tracers sensitive to different star formation epochs and.  

\section{Research Statement}
The extreme brightness of SNe~Ia are standardized using correlations between peak brightness and both SN~Ia `blueness' and explosion duration to reduce the already naturally low scatter in peak brightness, allowing their use as cosmic `rulers' to constrain cosmological models and explore the properties of dark energy \citep{Perlmutter1999}.
A lack of theoretical understanding SN~Ia necessitates these transient objects be parameterized by generative empirical models trained from representative SN samples.
The current linear Tripp standardization technique and commonly used SN~Ia models cannot differentiate between time-independent wavelength (color) variation intrinsic to the SN~Ia population and color variation from host galaxy dust attenuation \citep{Mandel2017}, and overcorrect for the established SN~Ia host bias where brighter SN~Ia prefer younger, less massive hosts with higher star formation rates (SFR) and younger stellar populations \citep{Sullivan2010,Rigault2018}.
SFR and stellar age are highly correlated with both each other and dust attenuation
This host bias propagating through Tripp standardization increases cosmological measurement uncertainty and is driven both by mishandling of dust effects on SN~Ia color variation and ignoring the established host bias during standardization \citep{Brout2021,Rose2021}.

\subsection{Determining the Connection between Dust and the Host Bias}
%{\bf Determining the Connection between Dust and the Host Bias:}
My first paper, recently resubmitted after review, compared observation and fitting techniques used when measuring the SN~Ia host bias. The first paper alleviated concern that such choice in technique biased mass step measurements, but it remains unclear if this is true for SFR and stellar age.  The first of two current projects is quantifying dependence of host SFR and stellar~age on host dust attenuation measurement using the observational SFR tracers H$\alpha$ emission, which provides an instantaneous ($<10$~Myr) SFR measurement, and UV flux, which provides an extended ($10-100$~Myr) SFR measurement.
UV and H$\alpha$ vary dramatically in their susceptibility to dust attenuation, providing independent probes of host dust content.
Our planned integration of sophisticated galaxy attenuation curve modeling from~\cite{Salim2018} and~\cite{Narayanan2018} will be a dramatic improvement from past SN~Ia analyses and increase the quality of UV flux dust corrections.
Numerous covariates including latent model parameters intrinsic to SN~Ia standardization necessitates efficient posterior analysis which will be performed using a hierarchical Bayesian framework implemented with a Hamiltonian Monte Carlo sampler built into the software package Stan~\cite{STAN}.

\subsection{Separating Sources of SN~Ia Color Variation}
%{\bf Separating Sources of SN~Ia Color Variation:}
My second project's genesis is from a DOE SCGSR award I received in 2019 to work with Dr.~Alex Kim of the Lawrence Berkeley National Laboratory using spectroscopic SN~Ia time series to build a hierarchical Bayesian model with Gaussian processes to agnostically disentangle the intrinsic color variation of SNe~Ia from host dust effects.
Parameterizing color variation using a set of multiplicative color variation curves that makes no strong prior assumption on color curve shape, our model exploits mathematical symmetries to simultaneously recover dust attenuation effects and intrinsic color variation.
Already, we have developed the mathematical and software foundation for this complex model and have found clear evidence for intrinsic color variation independent from variation due to dust.
We are preparing to write our first paper presenting our initial results.
Our plan over the next year is to fully develop our software into a fully descriptive SN~Ia empirical model by adding parameterized time-dependent color variation templates.
Extending our near infrared photometric and spectral time series to better constrain dust attenuation is an intermediate goal.
In the long term, this model will be promoted to a proper Gaussian mixture model to simultaneously fit light curves for all SN~Ia subtypes whilst acting as a classifier.\\

{\bf Mentoring Undergraduate Researchers:}
Joining the host bias analysis project are two undergraduates of the Michael Wood-Vasey research group who I will mentor during my remaining time as a PhD candidate.  This will include preparing to present their work to the Dark Energy Science collaboration and guidance in manuscript preparation and writing.  The Z. Daniels Fellowship would increase my available time to mentor these students. 

{\bf Dissertation Progress Notes:}
My second committee meeting and DOE award were both delayed by health issues during 2020 and the COVID-19 pandemic. 
The second meeting is now planned for June 2021 and the DOE award, restarted remotely in January 2021, will conclude July 2021. 

\newpage
\bibliographystyle{aasjournal}
\bibliography{main}

\end{document}
